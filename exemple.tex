\documentclass{article}
\usepackage[T1]{fontenc}
\usepackage[french]{babel}
\usepackage{utils}

\title{Macros générales \LaTeX}
\author{François Lareau\\OLST, Université de Montréal}
\date{\today}

\begin{document}
\maketitle

\section{Macros générales}

\begin{itemize}
	\item Guillemets de réserve: ce qu'on appelle des \scare{scare quotes}
	\item Emphase: pour mettre une plus forte \strong{emphase} sur un mot.
	\item Terme: pour mettre en évidence un nouveau \term{terme} dans le texte.
	\item Brouillon: \draft{notes rapides, à réécrire clean plus tard}
\end{itemize}


\section{Code}

On peut insérer des caractères accentués dans les extraits de code.

\begin{lstlisting}
text = "Ça marche avec les caractères français comme àéïôù"
for word in text.split():
  print(word)
\end{lstlisting}


\section{Figures}

Pour insérer une figure flottante externe, on peut utiliser la commande \lstinline{\fig}, qui accepte deux ou trois arguments:

\begin{enumerate}
	\item L'extension du fichier à insérer (cet argument est optionnel; par défaut sa valeur est \lstinline{pdf}).
	\item Le nom du fichier; cette chaîne de caractères sera aussi le label à utiliser pour les références (avec le préfixe \lstinline{fig:}).
	\item La légende.
\end{enumerate}

Par exemple, \lstinline|\fig[png]{image}{Une image bidon}| produit la figure~\ref{fig:image}.

\fig[png]{image}{Une image bidon}

Si la figure est plus large que la page, elle est automatiquement réduite. Par exemple, la figure~\ref{fig:graphe} a une largeur de plus de 18~cm, mais la commande \lstinline|\fig[png]{graphe}{Un graphe trop large}| en réduit la taille pour ne pas dépasser dans les marges.

\fig[png]{graphe}{Un graphe trop large}

Le fichier à insérer doit se trouver dans le répertoire \lstinline{figs}. Sinon, un message d'erreur s'imprime, comme à la figure~\ref{fig:ghost}.

\fig{ghost}{Une figure manquante}


\end{document}