\documentclass{article}
\usepackage{utils}

\title{Macros générales \LaTeX}
\author{François Lareau\\OLST, Université de Montréal}
\date{\today}

\begin{document}
\maketitle

\section{Macros générales}

\begin{itemize}
	\item Guillemets de réserve: ce qu'on appelle des \scare{scare quotes}
	\item Emphase: pour mettre une plus forte \strong{emphase} sur un mot.
	\item Terme: pour mettre en évidence un nouveau \term{terme} dans le texte.
	\item Brouillon: \draft{notes rapides, à réécrire clean plus tard}
\end{itemize}


\section{Code}

On peut insérer des caractères accentués dans les extraits de code.

\begin{lstlisting}
\begin{itemize}
	\item Guillemets de réserve: ce qu'on appelle des \scare{scare quotes}
	\item Emphase: pour mettre une plus forte \strong{emphase} sur un mot.
	\item Terme: pour mettre en évidence un nouveau \term{terme} dans le texte.
	\item Brouillon: \draft{notes rapides, à réécrire clean plus tard}
\end{itemize}
\end{lstlisting}


\section{Figures}

Pour insérer une figure externe, la commande \lstinline{\fig} est pratique. Elle demande trois arguments:

\begin{enumerate}
	\item L'extension du fichier à insérer (cet argument est optionnel; par défaut sa valeur est \lstinline{pdf}).
	\item Le nom du fichier; cette chaîne de caractères sera aussi le label à utiliser pour les références (avec le préfixe \lstinline{fig:}).
	\item La légende
\end{enumerate}

Par exemple, \lstinline|\fig[png]{test}{Une jolie figure}| produit ceci:

\fig[png]{test}{Une jolie figure}

Le fichier à insérer doit se trouver dans le répertoire \lstinline{figs}. Comme on peut voir à la figure~\ref{fig:test}, un message d'erreur s'affiche si le fichier n'existe pas.

\end{document}